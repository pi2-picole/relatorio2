\chapter{Introdução}

 No ano de 2015 a Organização das Nações Unidas (ONU) estimou um índice  de crescimento populacional que relatou a população mundial com 7,3 bilhões de habitantes na metade do ano de 2015, e  demonstrou, ainda, uma prospecção para o ano de 2030 que relata um aumento de 1,2 bilhões no número de habitantes, ou seja a população chegará a
8,5 bilhões \cite {ONU}. 

Aproveitando este contexto de crescimento populacional, tem-se a necessidade do aumento da produção de alimentos, uma vez que, tem-se maior demanda. Com isso, o setor de alimentos é o ramo que mais cresce no Brasil e no mundo. O ramo alimentício possui um acirramento da competição devido à grande segmentação da indústria de alimentos. Isso implica que o uso de tecnologias facilita, aprimora e diversifica o mercado. \cite{IBIA}

A proposta do projeto agrega tecnologia e inovação por meio de uma máquina do tipo \textit{Vending Machine} para picolés que pode ser deslocada com facilidade, é alimentada por um conjunto de placas solares e bateria, possui \textit{software} para vendas online, sistema de controle e automação integrados.

\section{Problemática}

Segundo uma consultoria realizada pelos membros da equipe com a empresária Uilma Ribeiro, proprietária da indústria Saborizze®, um freezer comum da Mercofricon 8/A possui 400W de potência e consumo de energia de 3,8kW/24h e faixa de tensão de 220V/60Hz e este consome muito para congelar e manter picolé. Quando se trata do congelamento de placas que mantém o carrinho de picolé referente a figura \ref{fig:carrinho}, este freezer consome ainda mais, uma vez este deve recongelar a placa que descongelou durante o dia na venda, isto aumenta os gastos da indústria distribuidora. Ademais, a empresa deve pagar por custos com funcionários para distribuição e para vendas. 

\begin{figure}[htb]
	\centering
    \includegraphics[scale=0.3]{figuras/carrinho}
    \caption{Carrinho de picolé da indústria Saborizze}
    \label{fig:carrinho}
\end{figure}


Visando um diferencial no método de vendas para o mercado competitivo atual no setor alimentício, o projeto $\pi col\acute{e}$ atenderá de forma automática o cliente permitindo acesso prático, metodologia sustentável, diversificação de público e distribuição rápida.

O problema em conjunto com suas causas, estão representados em um diagrama do tipo \textit{FishBone}, também conhecido como diagrama de causa e efeito, o qual é demonstrado na Figura \ref{fig:fishbone}. 

\begin{figure}[htb]
	\centering
    \includegraphics[keepaspectratio=true,scale=0.4]{figuras/fishbone}
    \caption{Fishbone}
    \label{fig:fishbone}
\end{figure}


Para uma descrição sucinta do problema, utilizou-se o seguinte \textit{framework}:

\textbf{O problema:} Gasto excessivo com funcionários e energia.

\textbf{Afeta:} Indústrias de sorvete e sorveteria.

\textbf{Cujo impacto é:} Impacto ambiental devido ao consumo de energia, aumento do preço do produto e diminuição da qualidade do produto final.

\textbf{Uma solução bem sucedida seria:} A criação de uma Vending Machine, a qual automatiza a venda dos picolés sendo possível colocá-la em diversos pontos de uma cidade.

\section{Estado da Arte}

Existem diversos estilos de venda, como direta, corporativa, casada, consignada e consultiva. Considerando o tipo de venda direta, o qual é possível aumentar as vendas, aborda-se um sistema de comercialização de bens de consumo e serviço singular, por meio do contato entre que acontece entre vendedores e compradores sem a necessidade de um estabelecimento físico ou mesmo fixo. \cite {MARQUES} 

	Analisando os tipos de venda, e aderindo ao método direto, a utilização de uma máquina do estilo \textit{Vending Machine} se destaca para o projeto $\pi col\acute{e}$. Uma \textit{Vending Machine} é uma máquina automática que oferece diversos tipos de produtos como cafés, doces, salgados e bebidas sem contato manual para a liberação dos consumíveis, ou seja, são máquinas de autosserviço. Esse tipo de máquina pode ser observada na figura \ref{fig:vending_machine}.
	Esse tipo de \textit{Vending Machine} é normalmente alimentada pela rede em 220V, com uma interface simples e direta com o usuário, apresenta normalmente um consumo elevado de energia, quando acompanhada de um sistema de refrigeração para venda de produtos gelados.

\begin{figure}[H]
	\centering
    \includegraphics[scale=0.8]{figuras/vending_machine}
    \caption{Vending Machine $\pi col\acute{e}$}
    \label{fig:vending_machine}
\end{figure}

Segundo o manual da ROBO QUENCHER as especificações de potência, dimensões e refrigeração são especificados, sendo assim, para 120v,60 hz, 11amps, 1320watts e para  230v, 50hz, 5amps, 1150watts de potência, com altura de 1830mm, profundidade de 864mm e largura de 1130mm de dimensões e um Compressor de 2 Cavalos, usando um fluido refrigerante R134A, peso de 0,37 kg, e pressões do projeto no lado alto de 200 psi e no lado baixo de 135 psi. A partir dessas informções e dentre outras pesquisadas o $\pi col\acute{e}$ apresenta funcionalidades semelhates de acordo com a especificação do projeto, acrescentando aspectos como vendas por cartão de crédito por meio do \textit{webapp}, controle de estoque, mecanismo de liberação por espiral, relatório de valores aferidos de temperatura e controle, alimentação por painéis fotovoltaicos, autonomia de seis à dez horas, estrutura diferenciada e inovadora \cite{ROBOQUENCHER}.



Além disso, utiliza-se uma logo desenvolvida pela equipe de projeto para influenciar na venda do protótipo para indústrias de sorvete e sorveterias. A logo é demonstrada na figura \ref{fig:logopicole}.

\begin{figure}[H]
	\centering
    \includegraphics[scale=0.2]{figuras/logo_picole}
    \caption{Logo do projeto $\pi col\acute{e}$}
    \label{fig:logopicole}
\end{figure}

As cores escolhidas remetem a sustentabilidade, a alegria e praticidade oferecidas pelo protótipo do projeto. 

\section{Justificativa}

Um sistema que conquiste o cliente, seja prático, atrativo, sustentável, seguro e tecnológico é a solução para o mercado competitivo atual.
A máquina de vendas além de reduzir custos com desperdício de recursos humanos, visa melhorar significativamente as condições de armazenamento do produto provendo, assim, um aumento das vendas e uma redução de custos a longo prazo.
O protótipo deste projeto é uma inovação no mercado, pois ainda não existe um estilo de máquina de vendas capaz de ser autônoma de seis à dez horas sem precisar de carga e ainda de fácil transporte. 
Dentre as soluções encontradas e estudadas, o protótipo apresenta chances de mercado, segundo pesquisa realizada com a indústria \textit{Saborizze}®, e também viabilidade financeira e comercial.

\section{Objetivos}

\subsection{Geral}

O objetivo do projeto é construir uma \textit{Vending Machine} que receba o cliente quando estiver perto, dispor de um \textit{Webapp} capaz de realizar vendas por cartão, controlar estoque, mecanismo de liberação de produto e relatório dos valores aferidos de temperatura e controle.  

\subsection{Específicos}

\begin{itemize}
\item Garantir a integração de todas as engenharias;
\item Garantir que todos os integrantes estejam engajados no projeto;
\item Projetar uma estrutura que não danifique o freezer muitos menos permita o tombamento do mesmo;
\item Validar com segurança a venda autônoma de picolés;
\item Obter um sistema de segurança contra furto eficaz;
\item Receber o cliente de forma agradável;
\item Chamar atenção do consumidor;
\item Certificar-se que a temperatura média do freezer seja -5º Celsius com margem de erro de no máximo 5º pra mais;
\item O veículo deverá ser capaz de manter-se resfriado por um período de aproximadamente 8 horas;
\item O sistema de refrigeração deverá ser totalmente vedado;
\item Utilizar um sistema de \textit{software} para vendas, controle e relatórios.
\end{itemize}


\section{Escopo}
A fim de contemplar a adversidade identificada e os objetivos explanados pelo grupo, foi definido o escopo geral do projeto. A proposta será atestada por meio do desenvolvimento do produto mínimo viável, utilizando o mesmo como objeto de estudo para obter a solução da problemática envolvida neste trabalho. O produto mínimo contempla então, para fins de prototipação, uma máquina de vendas com as seguintes características:

\begin{itemize}
\item Capacidade de realizar a venda de picolés de forma autonoma;
\item Capacidade de se resfriar para manter os picolés na temperatura adequada;
\item Capacidade de se manter \textit{off-grid} durante o período de venda;
\item Capacidade de receber pedidos de compra realizados por meio de um \textit{web app};
\item Capacidade de efetuar diversas medições da temperatura e umidade do ar em intervalos de tempo predefinidos;
\item Sistema vinculado capaz de receber os dados em um arquivo e disponibilizar de forma gráfica o relatório de vendas;
\end{itemize}


