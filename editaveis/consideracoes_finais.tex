\chapter{Considerações Finais}


Ser inovador e útil nos dias de hoje é um desafio que todo engenheiro enfrenta. Inovar por inovar, sem agregar nada a sociedade, não é algo que deve ser incentivado. A inovação tem que ir de encontro com as necessidades reais. Pensando nos problemas já elucidados, como aumento populacional, gasto excessivo com energia e dificuldades no atendimento ao consumidor, por exemplo, surgiu a  proposta de construir uma  \textit{vending machine} de picolés.  A ideia é implementar uma máquina que facilite a vida de consumidores de picolés, de vendedores  e também de empresários do ramo. O projeto visa diminuir custos, aumentar as vendas e trazer comodidade e praticidade ao consumidor. 

Para que os objetivos sejam alcançados foi realizado um planejamento sobre cada etapa do processo de construção do produto. Sendo assim, o problema foi encontrado, a solução foi discutida e proposta, as responsabilidades foram distribuídas, o cronograma de atividades definido, os sistemas e subsistemas foram apresentados, os riscos foram levantados e o custo foi calculado. Sem esquecer da gestão e comunicação, partes essenciais para que um projeto não saia dos trilhos. 

Fez-se vendas de picolés na FGA para arredação de fundos para o projeto, demonstrada no apendice \ref{app:pessoas}. Apresentou-se aqui etapas do projeto e também partes da solução física pronta, validações dos sistemas e testes realizados de cada subsistema. Espera-se para a etapa final do projeto as validações finais de cada subsistema e também a integração total dos mesmos.

Os trabalhos desenvolvidos durante o projeto demonstrou integração e cooperação de todos os membros do grupo. Houveram discussões sadias sobre os problemas e desafios encontrados, bem como sobre as soluções disponíveis. Acrescenta-se também a cooperação de todos os subgrupos com outros subgrupos do projeto, o que demonstra que a equipe está bem engajada e animada com o projeto.






