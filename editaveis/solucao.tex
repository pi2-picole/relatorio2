\chapter{Solução}

\section{Solução Geral}

A solução encontrada pela equipe foi de modernizar a compra de picolés, tornando o processo mais rápido, acessível e de maior qualidade. A equipe de software construirá um sistema de comunicação através de um WebService, e este WebService se comunicará com dois tipos distintos de usuários: usuários clientes e usuário vendedor, cada qual com acesso a informações pertinentes que serão gerados pelo aplicativo.

A parte eletrônica da máquina de vendas estará integrada a um sistema de localização GPS e contará com um sistema de segurança com alarmes sonoros capazes de alertar o vendedor de algum furto ou chamar a atenção dos clientes próximos. O sistema de sensoriamento será totalmente integrado ao sistema de refrigeração para manter a temperatura interna numa faixa ótima de operação e evitar o degelo dos picolés. 

A entrega de picolés ao consumidor final funcionará através de um sistema de molas integrado a um servomotor. O compartimento de armazenamento dos produtos e dos equipamentos será construído e adaptado a fim de se obter uma máquina leve, pequena e que dificulte a troca de calor com o meio externo, tudo integrado abaixo de uma estrutura de sustentação de painéis solares, que realizarão o fornecimento de energia para refrigeração dos picolés.

A refrigeração dos picolés será realizada através de um compressor instalado no fundo da caixa térmica,  juntamente com o banco de baterias que alimentará os equipamentos eletrônicos. Para facilitar o transporte e manuseio da máquina serão instalados sistemas de acoplamento para transporte e um sistema de interface visual para os usuários clientes terem maior facilidade na adaptação ao novo tipo de vendas de picolés.

A seguir detalhou-se as soluções de cada sub área.


\section{Solução de Software}

O subsistema de software será composto por um  Webapp, dividido entre funções para para o cliente e para o vendedor. Esse webapp fará comunicação com um webservice, construído pela equipe de software.

	O seção no web app para o vendedor permite a a manutenção do estoque de picolés. Ele poderá cadastrar diferentes sabores em diferentes quantidades, a qualquer momento. 
%     Além disso o vendedor também poderá enviar uma notificação, compartilhando sua localização com todos os clientes próximos. Eles, então, poderão traçar uma rota para comprar o picolé. 
    Outra funcionalidade, é de ler relatórios sobre vendas e formas de pagamento.
    
    A seção no web app para o cliente terá a função de compra. O cliente em posse do webapp pode selecionar o vendedor e sabor do picolé que deseja comprar. Após determinado a quantidade dos sabores desejado o cliente irá inserir o número do cartão de crédito e efetuar a compra. Após confirmação do pagamento, o aplicativo libera um botão, o qual será responsável pela liberação do picolé. Com o botão liberado, o cliente se dirige até a máquina de vendas em questão no momento que desejar e pressiona o botão, com isso, os picolés comprados são liberados e o usuário pode retirar seu pedido. Além disso, o cliente poderá traçar uma rota para a máquina de vendas que escolher.

\begin{itemize}
	\item Formas de pagamento:
    \begin{itemize}
        \item \textbf{Cartão:} Integração com um gateway de pagamento, para possibilitar o pagamento em cartão. O webapp não armazenará os dados de cartão do cliente em hipótese alguma.
  	\end{itemize}

	\item Integração com outros serviços:
  	\begin{itemize}
    	\item \textbf{Localização:} Vendedor deve habilitar o uso da localização
    	\item \textbf{Maps:} Cliente pode ver o vendedor mais próximo e traçar rota até ele.
  	\end{itemize}

	\item Relatórios:
  	\begin{itemize}
        \item Quantidade de vendas: por período/por sabor
        \item Variação da temperatura interna do refrigerador (sensor de temperatura)
        \item Proporção de vendas por modo de pagamento
  	\end{itemize}
\end{itemize}


\section{Solução de Eletrônica}

A parte eletrônica da máquina de vendas de picolé é dividida nos seguintes sistemas:

\subsection{Sistema GPS e Segurança}

A máquina de vendas terá um sistema de segurança utilizando-se um sistema de posicionamento global (GPS). O dispositivo GPS utilizado é o módulo da família de GPS U-blox stand-alone com chip GY-NEO6MV2 integrado. A plataforma escolhida para adquirir os dados de posicionamento foi o Arduino devido à facilidade de implementação com bibliotecas já prontas. A seguir, a foto do módulo:

\begin{figure}[H]
	\centering
    \includegraphics[width=0.5\textwidth]{figuras/modulo_gps}
    \caption{Módulo GPS da família U-blox}
    \label{fig:GPS}
\end{figure}

\subsubsection{Implementação}

Para a comunicação do módulo integrado do GPS é necessário o uso de duas bibliotecas. A primeira delas é a SoftwareSerial.h, a qual permite a comunicação serial em outros pinos digitais do Arduíno utilizando o software.  uma vez que é necessário uma comunicação serial por software. Já a biblioteca TinyGps.h fornece as funcionalidades como posição e curso com códigos mais simples, ela também mantém o recurso em consumo baixo e evita a dependência de pontos flutuantes obrigatórios.

\subsubsection{Teste}


\subsection{Sistema de Som e Alarme}

Para o sistema de segurança integrado ao projeto, é necessário emitir um aviso caso a máquina de vendas seja violado ou removido do local designado pelo vendedor. Além disso, há um assistente com voz automática para receber um cliente próximo à máquina de vendas e, quando não houver nenhuma pessoa por perto, o sistema emite uma propaganda do produto. O sistema de som é responsável por atender os clientes quando estes se aproximam, fazer propaganda e alarmar em caso de violação.

O circuito amplificador montado foi utilizando o circuito integrado TDA8571, o qual é um amplificador dedicado capaz de oferecer 40 W para até 4 alto falantes de 4 ohms cada.

\subsubsection{Implementação}

O esquemático do circuito montado foi retirado do Datasheet \cite{mq2} e está esquematizado a seguir:

\begin{figure}[H]
	\centering
    \includegraphics[width=0.5\textwidth]{figuras/circuito_TDA}
    \caption{Esquemático utilizado para o circuito do amplificador de som}
    \label{fig:circuito_TDA}
\end{figure}

O circuito integrado é constituído de 8 amplificadores ligados em ponte, como mostrado acima. Além disso, foi acrescentado um fusível de 8A e um diodo na entrada para proteção contra curto circuito e inversão de polaridade. Os pinos foram ligados conforme esquematizado a seguir:

\begin{itemize}
	\item \textbf{Pino 15:} É o pino de modo (mute e enable) que ativa ou desativa a saída. Foi ligado ao Raspberry para o controle de som.
    \item \textbf{Pinos 3, 6, 18, 21:} Alimentação negativa (terra 0 V) de cada um dos amplificadores.
    \item \textbf{Pinos 1, 8, 16, 23:} Alimentação positiva de cada um dos amplificadores. Foi utilizado 12 V, que é a tensão da fonte disponível do projeto.
    \item \textbf{Pinos 10, 11, 13, 14:} São as entradas de áudio. Como o áudio é o mesmo para todas as saídas, estes pinos foram ligados à mesma entrada.
    \item \textbf{Pino 2, 4, 5, 7, 17, 19, 20, 22:} São as saídas de áudio. Foram ligados 4 alto-falantes de .......
    
    \item \textbf{Pino 15:}
    \item \textbf{Pino 15:}
    \item \textbf{Pino 15:}
\end{itemize}

\subsection{Sistema de Sensoriamento}

O projeto exige duas medidas para garantir seu bom funcionamento: a temperatura interna do recipiente dos picolés em resfriamento e um detector de presença para atender um possível cliente e para o sistema de alarme. 

	Para a medida de temperatura, utiliza-se o sensor DS18B20 (que opera de 3.3 a 5 volts) à prova d'água fabricado pela Maxim Integrated, o qual é ideal para medidas em temperaturas na faixa de -10ºC a 85ºC com precisão de 0.5ºC, obtendo uma faixa de operação ainda maior. Para o controle de acionamento do sistema de refrigeração usufruiu-se da plataforma UART do Arduíno UNO. A figura \ref{fig:sensor_temperatura} ilustra o sensor.
    
\begin{figure}[H]
	\centering
    \includegraphics[width=0.5\textwidth]{figuras/sensor_temperatura}
    \caption{Sensor de temperatura}
    \label{fig:sensor_temperatura}
\end{figure}

O instrumento de sensoriamento de presença é o sensor ultrassônico de distância HC-SR04 (que opera a 5 volts), o qual será utilizado para detectar clientes nos quatro lados do recipiente de vendas (utilizando assim, quatro sensores) e para detectar a liberação de um picolé. O mesmo possui um alcance máximo de 4 metros e mínimo de 2 centímetros, sendo ideal para a detecção de usuários que se aproximam da máquina de vendas \cite{mq3}. A figura \ref{fig:sensor_presenca} ilustra o sensor.

\begin{figure}[H]
	\centering
    \includegraphics[width=0.4\textwidth]{figuras/sensor_presenca}
    \caption{Sensor de presença}
    \label{fig:sensor_presenca}
\end{figure}

\subsubsection{Implementação}

\subsubsection{Testes}

\subsection{Sistema dos Motores}
Os picolés serão entregues por meio de um sistemas de molas similar a de máquinas de venda. Um picolé situado no meio de uma mola faz um movimento linear quando este se rotaciona, como ilustra a imagem \ref{fig:sistema_molas} a seguir:

\begin{figure}[H]
	\centering
    \includegraphics[width=0.8\textwidth]{figuras/sistema_molas}
    \caption{Sistema de molas capaz de movimentar os picolés}
    \label{fig:sistema_molas}
\end{figure}

Logo, é necessário inserir um motor ao final de cada mola e fazer o controle do giro para que os picolés caíam conforme o pedido do cliente. Para que se realizasse este objetivo, a proposta inicial se tratava da utilização de um motor do tipo servomotor, uma vez que teoricamente, este, possui o torque suficiente para movimentar a longa fileira de picolés situados na mola com uma precisão de uma volta (360\degree por venda). Uma outra justificava a facilidade de controle com apenas um fio, contudo na prática e durante os testes realizados notou-se uma imprecisão no giro quando se retirava a trava e alterava o potênciometro original por um resistor para que, o mesmo, conseguisse dar um volta completa de 360\degree. Devido a esses problemas, a nova escolha se trata de um motor encontrado em sucata. Este é um motor de passos TAMAGAWA SEIKIM do tipo TS3103N145, com 200 passos por revolução e torque de 6 Kgfcm, além de ser DC e com alimentação de 12V.
Sabe-se que o motor de passos necessita de um driver, e para isso, montou-se o drive de controle constituído por resistores, transístores e diodos.

\subsection{Sistema do Acionamento do Compressor}

A refrigeração do recipiente vai ser feita por meio de um compressor. Mantê-lo ligado o tempo todo não é viável devido à limitação da demanda energética. Para isso, será feito um controle de acionamento do compressor de acordo com os dados enviados pelo sensor de temperatura. Para ligar e desligar o compressor, será utilizado um módulo relé JQC-3F. O relé é um componente eletro mecânico capaz de realizar a função de uma espécie de chave selecionadora. Por meio dela, é possível fazer o acionamento de sistemas de alta potência com um controlador (i.e. Raspberry Pi) sem danificá-lo e puxar muita corrente.

\subsection{Sistema de Interface Visual}

A ideia central da máquina de vendas é a sua capacidade de realizar vendas de picolé sem um vendedor presente pelo aplicativo de celular. Portanto, convém utilizar um sistema visual na qual poderá mostrar informações ao usuário tais como:

\begin{itemize}
  \item O estoque dos sabores dos picolés;
  \item Um rápido tutorial de como obter e utilizar o aplicativo;
  \item Caso o picolé seja vendido, informar ao usuário sua retirada;
  \item Fazer propaganda do produto.
\end{itemize}

Para isso, haverá uma tela LCD embutido no máquina de vendas de fácil visualização para os clientes.

\subsection{Integração de Todos os Sistemas Eletrônicos}

O controlador escolhido para integrar todos os sistemas será o Raspberry Pi 3, o qual é possui um computador que possui um processador ARM de 1.2 GHz 64 bit, 1 Gb de RAM e bluetooth. Ficou popular por ser intuitivo de utilizar, o que fez com que seja utilizado no mundo inteiro em diversos projetos eletrônicos. Possui capacidade de embutir um sistema operacional como o Linux, o que pode tornar o sistema mais estável e com menos memória de uso. Sua escolha se justifica pelo fato de conter todos os requisitos exigidos para cada sistema, tais como:

\begin{itemize}
  \item Possui entradas e saídas digitais as quais podem ser usadas para controle dos sistemas GPS e controle dos motores e controle do sensoriamento;
  \item Possui saída de áudio para o sistema de som; 
  \item Possui saída de vídeo para o sistema de interface com o usuário.
\end{itemize}

Além disso, depois de projetado e testado cada sistema, será confeccionado no final uma \textbf{placa de circuito impresso (PCI)} para conter todos os circuitos de forma compacta, já que o espaço do projeto é limitado.

\subsection{Integração entre Software e Hardware}
A comunicação entre software e hardware será feita via internet e, para isso, será utilizado o módulo GSM GPRS SIM900 fabricado pela SIMCom. O módulo interligado à Raspberry PI 3 utilizará um cartão SIM para conexão com a internet e, dessa forma, a comunicação com o sistema de software será estabelecida.

\begin{figure}[H]
	\centering
    \includegraphics[scale=0.25]{figuras/modulo_gprs}
    \caption{Módulo GPRS SIM900 (SIMCom)}
    \label{fig:modulo_gprs}
\end{figure}

\section{Solução de Estrutura}

\subsection{Propostas de possíveis soluções}
Como solução para a estrutura foram levantadas várias possibilidades de produtos para realizar a venda dos picolés, afim de escolher um conjunto que atenda às necessidades da indústria de picolés. À principio as opções eram:

\begin{itemize}
\item Triciclo com compartimento dianteiro
\item Triciclo com compartimento traseiro
\item Máquina de venda estática
\end{itemize} 

\begin{figure}[H]
	\centering
    \includegraphics[width=0.6\textwidth]{figuras/exemplo2}
    \caption{Triciclo com Compartimento Dianteiro}
    \label{fig:exemplo2}
\end{figure}

\begin{figure}[H]
	\centering
    \includegraphics[width=0.6\textwidth]{figuras/exemplo}
    \caption{Triciclo com Compartimento Traseiro}
    \label{fig:exemplo}
\end{figure}

\begin{figure}[H]
	\centering
    \includegraphics[width=0.6\textwidth]{figuras/machinewrapsheader}
    \caption{\textit{Vending Machine}}
    \label{fig:machinewrapsheader}
\end{figure}

Após consultoria com a empresa de sorvetes Saborizze®, não se tornou viável a utilização de triciclos no transporte da máquina de vendas automáticas devido a necessidade de um funcionário da empresa contratado apenas para conduzir o triciclo. Vendo isso, como solução de projeto de acordo com a aplicabilidade no mercado, foi escolhido como estrutura de projeto uma máquina de venda automática (\textit{vending machine}) que pode ser carregada por carro de transporte até o ponto de interesse pra ser realizada a venda dos picolés.

\subsection{Proposta acolhida para o projeto}

Como dito anteriormente, foi escolhido uma estrutura estática com acoplamento para um mecanismo de transporte. A construção desta estrutura escolhida pôde então ser subdividida da seguinte maneira:
 
 \begin{itemize}
\item Estrutura de suporte das placas solares

\item Mecanismos para venda automatizada de picolé

O mecanismo para entrega dos picolés será composto por quatro espirais em metal encaixadas dentro do freezer e controladas por servo motores. Entre as espirais será construída uma parede em poliestireno para manter os picolés alinhados. Ao girar a espiral, os picolés serão empurrados na direção de uma cavidade que será feita no fundo do freezer, derrubando o picolé dentro de uma caixa que também possuirá isolamento térmico e será onde o cliente terá acesso ao produto. 

   \begin{figure}[H]
	\centering
    \includegraphics[width=0.8\textwidth]{figuras/cad_vista_lateral}
    \caption{CAD do sistema de liberação do picolé e compartimentos da máquina de vendas}
    \label{fig:cad_vista_lateral}
\end{figure}

\item Compartimentos da máquina de vendas

A máquina de vendas devera possuir três compartimentos principais: compartimento para baterias e compressor, compartimento do freezer com isolamento térmico e compartimento para os componentes eletrônicos. Na Figura abaixo podemos visualizar a parte da estrutura que será construída em madeira e possui o compartimento para baterias e compressor, compartimento para os motores e um encaixe superior para o freezer. 

   \begin{figure}[H]
	\centering
    \includegraphics[width=0.4\textwidth]{figuras/estruturabase}
    \caption{Estrutura em madeira}
    \label{fig:estruturabase}
\end{figure}

O freezer será adaptado a partir de uma estrutura doada, apresentada a seguir, onde será construído um isolamento térmico com placas de isopor e poliestireno. 

   \begin{figure}[H]
	\centering
    \includegraphics[width=0.4\textwidth]{figuras/freezer}
    \caption{Estrutura base do freezer}
    \label{fig:freezer}
\end{figure}

\item Carrinho para o transporte da máquina de vendas

	Será montada uma estrutura para facilitar o transporte da máquina de vendas. Foi optado por uma estrutura simples e que pudesse ser acoplada à máquina durante seu transporte e depois desacoplada para assim poder ser reutilizada no transporte de outra máquina de vendas. 
	Para a montagem da estrutura de transporte da máquina, será feita a adaptação em uma estrutura já pronta mostrada na imagem abaixo:
    
   \begin{figure}[H]
	\centering
    \includegraphics[width=0.4\textwidth]{figuras/Render_-_Carro_de_Transporte}
    \caption{Estrutura base para o transporte da máquina de vendas}
    \label{fig:Render_-_Carro_de_Transporte}
\end{figure}

\end{itemize} 

\section{Solução de Energia}

Os subsistemas de energia estão divididos conforme a organização desta secção. 

\subsection{Refrigeração}
Atendendo aos requisitos definidos para o projeto, o recipiente no qual os picolés são armazenados deverá manter-se refrigerado durante o período de venda. O processo de refrigeração consiste em retirar calor de uma corpo ou espaço para reduzir sua temperatura e transferir esse calor para outro corpo ou espaço \cite{campos2010refrigeraccao}. Portanto, a partir da análise das características da demanda de refrigeração, custos limitados e o tempo hábil para a construção e integração do produto em questão, encontrou-se duas opções de resfriamento:

\begin{itemize}
\item Células de Peltier;
\end{itemize}

\begin{itemize}
\item Compressor;
\end{itemize}


O módulo de Peltier é a maneira mais prática de se utilizar o efeito peltier em larga escala, e consiste em um arranjo de pequenos blocos de \textit{telureto de bismuto - $Bi_{2}Te_{3}$} dopados tipo N e tipo P montados alternamente e eletricamente em série entre duas placas de cerâmica com alta condutividade térmica.Os semicondutores utilizados possuem altos coeficientes de Seebeck e podem ser tipo N $(\alpha \leq 0)$ e tipo P $(\alpha \geq 0)$  \cite{campos2010refrigeraccao}.

A utilização dos módulos de peltier tem as seguintes vantagens:

-Não utiliza partes mecânicas móveis para refrigeração;

-Aquece e resfria dependendo apenas da polaridade da alimentação;

-Dispensa o uso de gases refrigerantes, tecnologia 100 por cento estado sólido;

-Funcionam em qualquer orientação com/sem gravidade diferente dos refrigeradores baseados em compressores.


O resfriamento utilizando compressores ocorre por meio da compressão mecânica de vapor de um fluido refrigerante. Fluido refrigerante é uma substância que, ao circular dentro de um circuito fechado, é capaz de retirar calor de um meio enquanto se vaporiza a baixa pressão \cite{teixeiraconcepccao}.

Ao se optar pelo resfriamento por compressão tem-se as seguintes vantagens:

\begin{itemize}
\item Baixo consumo;

\item Maior Eficiência;
\end{itemize}


Outra opção a ser utilizada no projeto é o resfriamento por meio de compressor, uma vez que o consumo da célula de peltier é extremamente alto, 70W para a placa de 40x40x4 milímetros. Para resfriar o recipiente seriam necessários entre 7 a 10 módulos, o que necessitaria uma fonte de 490W a 700W e um inversor. A fonte de fornecimento escolhida foi placas fotovoltaicas e para conseguir manter esse fornecimento seria necessário uma área para absorção de energia solar muito maior do que seria possível instalar na máquina de vendas de picolé. 


Para o dimensionamento do compressor, é necessário calcular a potência necessária para fazer o fluido refrigerante circular pela serpentina. Essa potência foi encontrada a
partir da formulação:

\begin{math} W_{c} = m_{f} * (h_{2} - h_{1})  \end{math}

Onde:

$W_{c}$ é a potência teórica do compressor (kJ/h);

$m_{f}$ é o fluxo de massa refrigerante (kg/h);

$h_{2}$ é a entalpia no inicio da compressão (kJ/kg);

$h_{1}$ é a entalpia no final da compressão (kJ/kg);


\subsection{Alimentação}
	As fontes de energia que serão utilizadas para alimentar o sistema de refrigeração da máquina de picolé são painéis fotovoltaicos e baterias. O painel solar será dimensionado levando em conta a demanda energética a qual a mesma precisará suprir para manter o banco de baterias em condições de manter o refrigerador em operação durante o período solicitado, bem como os outros equipamentos consumidores e referentes às outras áreas.
    
\begin{itemize}
\item Bateria
\end{itemize}
	  
      A tomada de decisão para escolha de uma bateria deve considerar a tensão e a corrente.  \cite{de2011}. O cálculo de autonomia de uma bateria necessita de diversos dados. Existem 3 parâmetros os quais são considerados os principais para a a escolha de uma bateria: curva de descarga, capacidade de armazenamento e capacidade de descarga \cite{meggiolaro2006tutorial}. A curva de descarga é a relação do decaimento da tensão ao longo do consumo da capacidade nominal. A capacidade da bateria quantifica o tempo para que ocorra uma descarga total, medido em Ah (Ampér hora). A capacidade de descarga é a quantificação da entrega de carga sem que ocorram danos à bateria.
      As baterias de chumbo-ácido são muito utilizadas em sistemas onde a há correntes  elevadas, como motores de arranque e máquinas elétricas. É composta por Chumbo e Ácido Sulfúrico em concentrações calculadas que variam de 27 à 37 por cento \cite{de2005estudo}, podendo ser seladas ou não, possuindo vida útil de até 4 anos, apresentando assim uma boa relação de custo-benefício.
            
\begin{itemize}
\item Painel Fotovoltaico
\end{itemize}
		A tomada de decisão para escolha de um painel fotovoltaico deve considerar a demanda energética dos sistemas os quais serão alimentados, no caso da máquina de vendas especificamente, deverá carregar a bateria que alimentará o compressor, e deixa-la em condições de uso durante o período solicitado. A tensão gerada pela placa deverá ser elevada para atender às necessidades do compressor, para isso um inversor será instalado, bem como um controlador de carga. 
        A decisão de escolha do painel teve como variáveis a disponibilidade e custos de implementação, as quais foram limitantes e corroboraram para que um painel de 20W, disponibilizado por um integrante do grupo fosse escolhido, mas que funcionará de maneira análoga a um painel de grande porte, e ainda que de baixa potência,  auxiliará no processo de recarga da bateria.
        O painel da marca \textit{Yingli Solar - Power Your Life},  será utilizado e possui as seguintes especificações:
        
        \begin{table}[]
\centering
\caption{Especificações Painel \textit{Yingli Solar - Power Your Life} }
\label{Especificações Painel \textit{Yingli Solar - Power Your Life} }
\begin{tabular}{|l|l}
\cline{1-1}
\cellcolor[HTML]{FFFFFF}{\color[HTML]{000000} \textbf{Yingli Solar-YL020P-17B 1/7}} &                                    \\ \hline
\textbf{Potência {[}W{]}}                                                           & \multicolumn{1}{l|}{\textbf{20}}   \\ \hline
\textbf{Tensão {[}V{]}}                                                             & \multicolumn{1}{l|}{\textbf{16,6}} \\ \hline
\textbf{Corrente {[}A{]}}                                                           & \multicolumn{1}{l|}{\textbf{1,2}}  \\ \hline
\textbf{Tensão Circuito Aberto {[}V{]}}                                             & \multicolumn{1}{l|}{\textbf{21,4}} \\ \hline
\textbf{Tensão Máxima do Sistema {[}V{]}}                                           & \multicolumn{1}{l|}{\textbf{50}}   \\ \hline
\end{tabular}
\end{table}
        
        	De acordo com o selo PROCEL em parceria com o INMETRO, esse painel possui uma eficiência do tipo D, correspondente à 11,8\%, o que é compatível com outros painéis comerciais. 
            A área externa do módulo \textit{Yingli Solar-YL020P-17B 1/7} é de 0,17m^{2}, possui uma produção média mensal de energia de 2,50kWh/mês, esse valores são importantes para que seja observado que um painel de maior dimensão apresentaria um melhor comportamento quando associado na \textit{Vending Machine}, mas por viabilidade econômica não será possível a implementação de um painel maior.
            Abaixo, uma foto do painel que sera instalado.
            
                  \begin{figure}[H]
    \centering
    \includegraphics[width=0.7]{figuras/painel_solar01}
    \caption{Yingli Solar-YL020P-17B 1/7 }
    \label{fig:painel_solar01}
\end{figure}
        
        
\begin{itemize}
\item Decisão da Bateria - Veicular x Estacionária
\end{itemize}

		Uma das decisões voltadas ao fornecimento de energia do projeto foi quanto ao do tipo de bateria que iria auxiliar no funcionamento da \textit{Vending Machine}. Para isso, foram analisados os tipos de bateria existentes no mercado, como as baterias automotivas, como o próprio nome já diz, são desenvolvidas para o uso em automóveis, com uma vida útil estimada de aproximadamente 3 anos, e com capacidade de suportar correntes de pico elevadas por um curto período de tempo, e as baterias de característica estacionária, as quais são construídas com materiais mais nobres, visando aumentar a sua vida útil e eficiência, relacionado diretamente à sua finalidade de manter equipamentos em funcionamento pleno. 
    	Uma análise feita nas especificações técnicas e características de ambas as baterias, como suas respectivas curvas de descarga, quantidade de ciclos em situação de uso diário, bem como a capacidade de suportar uma corrente elevada por um período de tempo maior, quando comparada às automotivas, tornam a bateria estacionária para o uso de equipamentos\textit{ off-grid}, a mais indicada.
   		A \textit{Vending Machine}, por ser um dispositivo que armazena produtos perecíveis, possuir uma câmara fria, além de ser um produto que tem como pré-requisito uma baixa manutenção, onde a confiabilidade do equipamento é um ponto muito importante, a bateria estacionária é a mais indicada para atender à proposta do equipamento, pela sua segurança, confiabilidade e comportamento mais estável além de suportar melhor descargas mais profundas, quando comparado com a uma bateria automotiva.
        
        \begin{itemize}
\item Dimensionamento da Bateria
\end{itemize}

\begin{itemize}
\item Dimensionamento do Inversor
\end{itemize}
        
\begin{itemize}
\item Relação de Consumo Energético
\end{itemize}
		A \textit{Vending Machine} possui dispositivos consumidores de energia elétrica, e para o dimensionamento de um sistema eficiente e confiável, é preciso analisar a relação de consumo energético, tempo de funcionamento do equipamento e energia disponível na bateria. Para isso, foi feito um levantamento da potência consumida pelos principais equipamentos e relacionados com o tempo de operação da máquina.
        
        \begin{table}[]
\centering
\caption{Relação de Consumo}
\label{my-label}
\begin{tabular}{lllll}
           & Potência {[}W{]} & Corrente {[}A{]} & Tensão{[}V{]} &  \\
Compressor & 337W             & 1,53A            & 220V          &  \\
Eletrônica &                  &                  &               &  \\
Iluminação & 5W/m             & 0,41A            & 12V           & 
\end{tabular}
\end{table}
		
        Com uma relação dos principais consumidores de energia da \textit{Vending Machine} é possível ter uma média de consumo, e com isso relacionar ao tempo em que a bateria alimentará o sistema de forma segura e eficiente, atendendo às necessidades do produto proposto. O dimensionamento consiste em uma relação de consumo entre o produto e a bateria, levando em consideração a capacidade da bateria. 
        Abaixo, temos um gráfico do comportamento de descarga da bateria estacionária FREEDOM DF500 (40Ah), a qual foi escolhida por atender às necessidades e permitir o funcionamento por um período de \textbf{(MOSTRAR ESTIMATIVA DE TEMPO).}

      \begin{figure}[H]
    \centering
    \includegraphics[width=0.7]{figuras/curva_descarga}
    \caption{Curva de Descarga}
    \label{fig:curva_descarga}
\end{figure}

		Na curva de descarga apresentada acima é possível verificar o decaimento da Tensão x Tempo, conforme o tempo passa com o consumo indicado no gráfico. O comportamento inicial esta diretamente relacionado com a tensão inicial, onde nota-se uma queda abrupta, seguida de uma estabilização e prolongamento até a descarga chegar em níveis mais próximos do limite inferior de 10,5V recomendados pelo fabricante, esse comportamento é característico das baterias estacionárias.
        Para a \textit{Vendind Machine}, a curva que mais se aproxima do comportamento esta entre \textbf{(PREECHER COM O VALOR DE CORRENTE TOTAL)}, com isso, o tempo de operação está de acordo com o que foi estimado inicialmente.
        
\begin{itemize}
\item Diagrama Simplificado
\end{itemize}

\begin{itemize}
\item Análise e Testes
\end{itemize}
			 Os testes são cruciais para a validação de todo e qualquer projeto, um dos primeiros testes feitos pelo grupo foi a confirmação do funcionamento do compressor e a verificação se o sistema de refrigeração, serpentinas e agregados estavam funcionando perfeitamente. Para isso, o conjunto foi desmontado e uma inspeção de todos os componentes foi feita. Com o compressor em funcionamento a validação de que o equipamento estava refrigerando foi feita através de um Termometro Infravermelho, modelo GM300 da Benetech, indicando a temperatura de -6,7\degree, o que de acordo com as especificações técnicas fornecidas pelo fabricante do equipamento. Para uma confirmação mais precisa do dado, uma segunda verificação foi feita com um Termopar da Minipa, o qual apresentou um resultado muito próximo do outro equipamento, o que confirma que o compressor está com um bom funcionamento e sua temperatura de operação atende às necessidades do projeto.
             
                   \begin{figure}[H]
    \centering
    \includegraphics[width=0.7]{figuras/medicao_laser}
    \caption{Medição GM300}
    \label{fig:medicao_laser}
\end{figure}                 

\begin{figure}[H]
    \centering
    \includegraphics[width=0.7]{figuras/medicao_termopar}
    \caption{Medição Termopar Minipa}
    \label{fig:medicao_termopar}
\end{figure}

		 Algumas baterias usadas foram testadas anteriormente à compra da FREEDOM DF500, para fins experimentais e para avaliação da possibilidade de uso no projeto. As baterias abaixo foram colocadas em um equipamento balanceador de células e carregador da marca \textit{Turnigy Reaktor - 250W - 10A}, objetivando verificar se as baterias estavam em condições de uso. O equipamento emite um registo das resistências internas das baterias, as quais nenhuma estava apta para atender às necessidades do projeto. Fato esse, que foi solucionado com a compra de uma bateria estacionária nova, a qual passou pelo mesmo equipamento e apresentou bons resultados.
         
  \begin{figure}[H]
    \centering
    \includegraphics[width=0.7]{figuras/baterias_testadas}
    \caption{Baterias Testadas}
    \label{fig:baterias_testadas}
\end{figure}
		
       O teste de funcionamento da bateria, inversor e compressor ligados em conjunto foi feito no Laboratório de Termoflúidos da UnB, inicialmente foram feitas medições de tensão de saída e entrada do inversor, com o propósito  de verificar se o funcionamento do equipamento esta de acordo com as especificações do fabricante, e para não danificar os subsistemas, a verificação da tensão da bateria para validação de que o produto estava com a tensão correta de operação. Todas essas medições foram feitas com um Multímetro Minipa ET-400. A montagem dos bornes das baterias e encaixes foram feitos de maneira a deixar a maior superfície de contato com as ponteiras dos equipamentos.
       O acionamento do sistema não acionou o compressor, e o Inversor entrou em modo de segurança\textbf{...(CONTINUAR)}
       
         \begin{figure}[H]
    \centering
    \includegraphics[width=0.7]{figuras/teste_bancada}
    \caption{Teste na Bancada - Laboratório UnB}
    \label{fig:teste_bancada}
\end{figure}